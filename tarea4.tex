\documentclass{article}

% formato
\usepackage[margin = 1.5cm, letterpaper]{geometry}
\usepackage[utf8]{inputenc}
\usepackage{graphicx}% http://ctan.org/pkg/graphicx
\usepackage{array}
\usepackage[table,xcdraw]{xcolor}

% autómatas
\usepackage{tikz}
\usetikzlibrary{automata, positioning}

%formato ecuaciones
\usepackage{amsmath}

% símbolos
\usepackage{amssymb}

% manejo de tablas
\usepackage{float}

\begin{document}
    \title{
        Lenguajes de Programación \\
        Ejercicio Semanal 4
    }

    \author{
        Sandra del Mar Soto Corderi \\
        Edgar Quiroz Castañeda
    }

    \date{
        5 de septiembre del 2019
    }
    
    \maketitle

    \begin{enumerate}
        \item {
            Considera las siguientes expresiones:\\
                            
            	$e_1 =_{def} \texttt{if iszero (2 + 5) then let x = 3 in suc (x + 1) end else pred (6 + 4)}$\\
            	
            	$e_2 =_{def} \texttt{let x = suc (2} \ast \texttt{1) in let y = pred x in if false then (x + y) else (0} \ast \texttt{ 5) end end}$\\
            
            %a)
            \begin{enumerate}
            	\item {
            	Verifica el tipado de $e_1$ y $e_2$, utilizando la semántica estática. Debe mostrar todos los pasos de derivación,        	anotando en cada uno de ellos el nombre de la regla utilizada.
            	
            	}
            	
            	%b
            	\item {
            	Realiza la evaluación formal de $e_1$ y $e_2$ utilizando la semántica operacional ($\rightarrow$). Debe mostrar todos los pasos de derivación, anotando en cada uno de ellos el nombre de la regla utilizada.
            	}
            \end{enumerate}
        }
    \end{enumerate}
\end{document}