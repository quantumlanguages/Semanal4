\documentclass{article}

% formato
\usepackage[margin = 1.5cm, letterpaper]{geometry}
\usepackage[utf8]{inputenc}
\usepackage{graphicx}% http://ctan.org/pkg/graphicx
\usepackage{array}
\usepackage[table,xcdraw]{xcolor}

% autómatas
\usepackage{tikz}
\usetikzlibrary{automata, positioning}

%formato ecuaciones
\usepackage{amsmath}

% símbolos
\usepackage{amssymb}

% manejo de tablas
\usepackage{float}

\begin{document}
    \title{
        Lenguajes de Programación \\
        Ejercicio Semanal 4
    }

    \author{
        Sandra del Mar Soto Corderi \\
        Edgar Quiroz Castañeda
    }

    \date{
        5 de septiembre del 2019
    }
    
    \maketitle

    \begin{enumerate}
        \item {
            Considera las siguientes expresiones:\\
                            
            	$e_1 =_{def} \texttt{if iszero (2 + 5) then let x = 3 in suc (x + 1) end else pred (6 + 4)}$\\
            	
            	$e_2 =_{def} \texttt{let x = suc (2} \ast \texttt{1) in let y = pred x in if false then (x + y) else (0} \ast \texttt{ 5) end end}$\\
            
            %a)
            \begin{enumerate}
            	\item {
            	Verifica el tipado de $e_1$ y $e_2$, utilizando la semántica estática. Debe mostrar todos los pasos de derivación,        	anotando en cada uno de ellos el nombre de la regla utilizada.\\
            	
            	Usaremos los nombres de las reglas mencionadas en las notas:
            	\begin{flalign*}
            	&e_1 = if\ iszero(2+5)\ then\ let \ x=3\  in\  suc(x+1)\  end\ else\ pred(6+4):Nat\\
            	1.\ & \vdash 3:Nat &\mathrm{(tnum)}\\
            	2.\ & \vdash 2:Nat &\mathrm{(tnum)}\\
            	3.\ & \vdash 5:Nat &\mathrm{(tnum)}\\
            	4.\ & \vdash 2+5:Nat &\mathrm{(tsum) 2,3}\\
            	5.\ & \vdash iszero(2+5):Bool &\mathrm{(tisz) 4}\\
            	6.\ & \vdash 6:Nat &\mathrm{(tnum)}\\
            	7.\ & \vdash 4:Nat &\mathrm{(tnum)}\\
            	8.\ & \vdash 6+4:Nat &\mathrm{(tsum) 6,7}\\
            	9.\ & \vdash pred(6+4):Nat &\mathrm{(tpred) 8}\\
            	10.\ & x:Nat \vdash x:Nat &\mathrm{(tvar)}\\
				11.\ & x:Nat \vdash 1:Nat &\mathrm{(tnum)}\\
				12.\ & x:Nat \vdash x+1:Nat &\mathrm{(tsum) 10,11}\\
				13.\ & x:Nat \vdash suc(x+1):Nat &\mathrm{(tsuc) 12}\\
				14.\ & \vdash let \ x=3\  in\  suc(x+1)\  end:Nat &\mathrm{(tlet) 1,13}\\
            	15.\ & \vdash if\ iszero(2+5)\ then\ let \ x=3\  in\  suc(x+1)\  end\ else\ pred(6+4):Nat &\mathrm{(tif) 5, 9, 14}
            	\end{flalign*}
            	
            	\begin{flalign*}
            	&e_2 = let\  x = suc (2 \ast 1)\  in\  let\  y = pred\  x\  in\  if\  false\  then\  (x + y)\  else\  (0 \ast 5)\  end\  end:Nat\\
            	1.\ & \vdash 2:Nat &\mathrm{(tnum)}\\
            	2.\ & \vdash 1:Nat &\mathrm{(tnum)}\\
            	3.\ & \vdash 2 \ast 1:Nat &\mathrm{(tprod) 1,2}\\
            	4.\ & \vdash suc(2 \ast 1):Nat &\mathrm{(tsuc) 3}\\
            	5.\ & x:Nat \vdash x:Nat:Nat &\mathrm{(tvar)}\\
            	6.\ & x:Nat \vdash pred\ x:Nat &\mathrm{(tpred) 5}\\
            	7.\ & x, y:Nat \vdash y:Nat &\mathrm{(tvar)}\\
            	8.\ & x, y:Nat \vdash x:Nat &\mathrm{(tvar)}\\
            	9.\ & x, y:Nat \vdash x+y:Nat &\mathrm{(tsum) 7,8}\\
            	10.\ & x, y:Nat \vdash bool[false]:Bool &\mathrm{(tfalse)}\\
            	11.\ & x, y:Nat \vdash 0:Nat &\mathrm{(tnum)}\\
            	12.\ & x, y:Nat \vdash 5:Nat &\mathrm{(tnum)}\\
            	13.\ & x, y:Nat \vdash 0 \ast 5:Nat &\mathrm{(tprod)11, 12}\\
            	14.\ & x, y:Nat \vdash if\  false\  then\  (x + y)\  else\  (0 \ast 5)\:Nat &\mathrm{(tif) 9,10,13}\\
            	15.\ & x:Nat \vdash let\  y = pred\  x\  in\  if\  false\  then\  (x + y)\  else\  (0 \ast 5)\  end:Nat &\mathrm{(tlet) 6, 14}\\
            	16.\ & \vdash let\  x = suc (2 \ast 1)\  in\  let\  y = pred\  x\  in\  if\  false\  then\  (x + y)\  else\  (0 \ast 5)\  end\  end:Nat &\mathrm{(tlet) 4, 15}
            	\end{flalign*}
            	
            	}
            	
            	%b
            	\item {
            	Realiza la evaluación formal de $e_1$ y $e_2$ utilizando la semántica operacional ($\rightarrow$). Debe mostrar todos los pasos de derivación, anotando en cada uno de ellos el nombre de la regla utilizada.
            	}
            \end{enumerate}
        }
    \end{enumerate}
\end{document}